\documentclass{article}
\usepackage[utf8]{inputenc}
\usepackage{amsmath}
\usepackage{amsfonts}
\usepackage{amssymb}

\begin{document}

\section*{Respuestas}

\begin{enumerate}
    \item Los diferentes tipos de flip-flops son:
    \begin{itemize}
        \item SR (Set-Reset): Controlado por entradas Set y Reset.
        \item D (Data): Transfiere el estado de la entrada de datos a la salida en el flanco del reloj.
        \item T (Toggle): Cambia de estado en cada pulso de reloj si la entrada está en alto.
        \item JK: Similar al SR pero evita la condición indeseada cuando ambas entradas son altas.
    \end{itemize}
    
    \item La elección del tipo de flip-flop depende de la función específica requerida, la simplicidad del diseño y la eliminación de condiciones indeseables.
    
    \item Para la representación de flip-flops SR y JK usando compuertas lógicas básicas, se detallarán los circuitos respectivos.
    
    \item Existen dos tipos de máquinas de estado finito:
    \begin{itemize}
        \item Mealy: La salida depende del estado actual y de las entradas.
        \item Moore: La salida depende solo del estado actual.
    \end{itemize}
    
    \item Los flip-flops de tipo JK son preferidos para realizar máquinas de estado finito debido a su versatilidad y capacidad para manejar condiciones indeseadas de manera eficiente.
    
    \item Un registro de desplazamiento que desplaza 1 bit a la izquierda se implementa conectando las salidas de los flip-flops a las entradas del siguiente flip-flop en la cadena. Para simular su funcionamiento en una ALU de 8 bits, se utilizaría una operación de desplazamiento hacia la izquierda.
\end{enumerate}

\end{document}
