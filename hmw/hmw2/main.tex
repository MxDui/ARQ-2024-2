\documentclass[12pt]{article}
\usepackage[utf8]{inputenc}
\usepackage[spanish]{babel}
\usepackage{hyperref}
\usepackage{natbib}

\title{Respuestas sobre Arquitectura de Computadoras}
\author{David Rivera Morales}

\begin{document}

\maketitle

\section*{Respuestas}

\subsection*{1. Arquitectura de Computadoras}
La Arquitectura de Computadoras no se limita únicamente al estudio de las instrucciones de una computadora y su desempeño respecto a estas. Incluye la organización lógica del hardware, cómo interactúan sus componentes y los principios que guían el diseño de procesadores, sus unidades y métodos de funcionamiento. \citep{profesional_review_2022}.

\subsection*{2. Registros}
Sí, los registros son dispositivos de hardware que permiten almacenar cualquier valor en binario. Son fundamentales para el funcionamiento de la CPU, ya que almacenan instrucciones, direcciones de memoria y datos necesarios para la ejecución de programas \citep{profesional_review_2022}.

\subsection*{3. Diferencia entre AMD Ryzen 5 e Intel Core i5}
Para obtener una comparación actualizada y detallada entre AMD Ryzen 5 e Intel Core i5, se recomienda consultar especificaciones técnicas recientes, ya que ambas líneas de procesadores evolucionan constantemente.

\subsection*{4. RISC vs. CISC}
La arquitectura RISC requiere un mayor número de instrucciones para realizar una tarea en comparación con CISC. Esto se debe a su enfoque en instrucciones más simples y específicas que permiten una ejecución más rápida y eficiente en algunos casos \citep{profesional_review_risc}.

\subsection*{5. Tres factores de desempeño}
Los tres factores de desempeño son: la cantidad de trabajo realizado por el procesador en un tiempo determinado, el tiempo que tarda en ejecutarse un programa, y la latencia o tiempo de respuesta del sistema. Estos dependen de la arquitectura del procesador, la eficiencia del código del programa y la velocidad de los dispositivos de E/S \citep{profesional_review_2022}.

\subsection*{6. Tiempo de CPU para 9 millones de ciclos a 3 ns}

\subsection*{7. Tiempo de CPU para 14 millones de ciclos a 2.4 GHz}

\subsection*{8. Periodo de señal de reloj en arquitecturas CISC y RISC}

\subsection*{9. Ley de Moore e Intel 4004}

\subsection*{10. Intel Core i9-9900K y Ley de Moore}

\bibliographystyle{apa}
\bibliography{references}

\end{document}

\section*{Bibliografía}
\begin{filecontents}{references.bib}
@misc{profesional_review_2022,
  title={Arquitectura de computadoras: ¿Qué son? ¿Cómo funcionan?},
  author={Profesional Review},
  year={2022},
  note={\url{https://www.profesionalreview.com/2022/10/01/arquitectura-de-computadoras-que-son-como-funcionan/}}
}

@misc{profesional_review_risc,
  title={RISC: La arquitectura de procesadores usada por ARM para cambiar el mercado},
  author={Profesional Review},
  year={2021},
  note={\url{https://www.profesionalreview.com/2021/07/17/risc-arquitectura-procesadores/}}
}
\end{filecontents}
